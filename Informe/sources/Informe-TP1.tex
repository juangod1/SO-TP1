\documentclass[10pt,a4paper]{report}
\usepackage[utf8]{inputenc}
\usepackage[spanish]{babel}
\usepackage{amsmath}
\usepackage{amsfonts}
\usepackage{amssymb}
\usepackage{fancyhdr}
\usepackage{color}
\usepackage{graphicx}
\usepackage{listings}
\graphicspath{ {../images/} }

\definecolor{darkcerulean}{rgb}{0.03, 0.27, 0.49}

\title{
	\bf\color{darkcerulean} Trabajo Práctico 1 \\
	\color{black}Sistemas Operativos (72.11) \\
	\includegraphics{itba-escudo2}
	}
\author{Juan Godfrid - 56609 \\Pablo Radnic - 57013 \\ Joaquín Ormachea - \\Francisco Delgado -}
\pagestyle{fancy}
\fancyhf{}
\lhead{ Instituto Tecnológico de Buenos Aires - 2018}

\fancyfoot[C]{\thepage}

\fancypagestyle{plain}{
	\fancyhf{}
	\fancyfoot[C]{\thepage}
	\renewcommand{\headrulewidth}{0.4pt}
	\renewcommand{\footrulewidth}{0.4pt}
	\lhead{ Instituto Tecnológico de Buenos Aires - 2018}
}

\begin{document}
\maketitle

\newpage
\tableofcontents
\newpage

%%%%%%%%%%%%%%%%%%%%%%%%%%%%%%%%%%%%%%%%%

\chapter{Diseño}
\section{Diseño del Sistema}
dshfdshfdshfds
\section{IPCs}
\subsection{Message Queue}
	El sistema utiliza dos \textit{message queues} del tipo POSIX (son más modernas que sus contrapartes SYSTEM V) en la intercomunicación de los procesos maestro y esclavo. El primero se utiliza para enviar los archivos que se requieren \textit{hashear} hacia los esclavos, mientras que el segundo se utiliza para enviar los \textit{hashes} ya procesados devuelta al proceso maestro. La utilidad de dicho IPC es una multicausalidad. Al ser una cola, el \textit{message queue} asegura el comportamiento LIFO, por lo que es ideal para enviar los \textit{hashes} de vuelta al proceso maestro, ya que asegura que el orden de llegada corresponde al orden de procesamiento. El \textit{message queue} también resulta adecuado para enviar los nombres de los archivos a los esclavos ya que los mensajes sirven como instrucciones atómicas: Los esclavos leen las instrucciones de la cola, cuando terminan, leen otra instrucción, y cuando está vacía la cola significa que el procesamiento concluyó. Todo esto se realiza sin gastar un solo ciclo de procesador del proceso maestro.
\subsection{Señales} 

%%%%%%%%%%%%%%%%%%%%%%%%%%%%%%%%%%%%%%%%%

\chapter{Utilización}
\section{Compilación y Ejecución}
El sistema posee un archivo de tipo \textit{GNU make} que discrimina en la compilación del código fuente en base a qué proceso debería pertenecer, de este modo logra compilar separadamente los archivos binarios de manera autónoma. El programa debe recibir como argumento por línea de comandos una lista de archivos a procesar, también puede recibir una expresión \textit{bash} para los archivos (por ejemplo: ./*.c). Un ejemplo de uso del programa es el siguiente:
\begin{lstlisting}
/SO-TP1$ make
/SO-TP1$ ./Binaries/run main.c
\end{lstlisting}


\section{Proceso Vista}
El \textit{make} compila también de manera separada el proceso vista, para ejecutarlo se debe enviar el siguiente comando:
\begin{lstlisting}
\end{lstlisting}

%%%%%%%%%%%%%%%%%%%%%%%%%%%%%%%%%%%%%%%%%

\chapter{Problemas y Resoluciones}
\section{Proceso Maestro}
asdf
\section{Proceso Esclavo}
sdfasdf
\section{Proceso Vista}
asdfadsf
Dummy text

%%%%%%%%%%%%%%%%%%%%%%%%%%%%%%%%%%%%%%%%%

\chapter{Limitaciones}
\section{Limitaciones del Sistema}
asdfasldjfasd

\section{Limitaciones de la Implementación}
asdfasdf
Dummy text

%%%%%%%%%%%%%%%%%%%%%%%%%%%%%%%%%%%%%%%%%

\chapter{Bibliografía}
\section{Extractos de código}

Dummy text

\end{document}